\documentclass[14pt]{report}

%Para ponerlo en español:
\usepackage[T1]{fontenc}
\usepackage[spanish]{babel}

%para poder citar
\usepackage[backend=biber]{biblatex}
\bibliography{bibliografía.bib}

%Para poder añadir imágenes:
\usepackage{graphicx}

% Title Page (ya no es necesario)
\title{Proyecto: Interfaz para un filtro de señal cardiaca}
\author{Caballero Trinidad Flor Isabel. \\ Cortés Ramírez Samuel Jefte. \\ García García Ian Pablo. \\ Pérez Roldan Eduardo Alejandro. \\ Cabañas Alba Alejandro. \\ Alfonso Gamboa Rubén.}


\begin{document}

%Portada
\begin{titlepage}
	\centering
	{\includegraphics[width=0.2\textwidth]{recursos/Universidad_Veracruzana}\par}
	\vspace{1cm}
	{\bfseries\LARGE Universidad Veracruzana \par}
	\vspace{1cm}
	{\scshape\Large Ingeniería Biomédica \par}
	\vspace{3cm}
	{\scshape\Huge Interfaz para un filtro de frecuencia cardiáca. \par}
	\vspace{2cm}
	{\itshape\Large Proyecto. \par}
	\vfill
	{\Large Autores: \par}
	{\Large Caballero Trinidad Flor Isabel. \\ Cortés Ramírez Samuel Jefte. \\ García García Ian Pablo. \\ Pérez Roldan Eduardo Alejandro. \\ Cabañas Alba Alejandro. \\ Alfonso Gamboa Rubén. \par}
	\vfill
	
\end{titlepage}

%\tableofcontents
\chapter{Parcial 1}
	\section{keywords}
		\begin{itemize}
			\item Frecuencia cardaica.
			\item Interfaz.
			\item Filtro de frecuencia cardaiaca.
			\item Interfaz para un filtro.
			\item SPython.
			\item Interfaz Médica.
			\item Ingeniería Biomédica.
		\end{itemize}
	
	\section{Introducción.}
		\subsection*{Pregunta de investigación.}
		¿Cómo desarrollar una interfaz que facilite la visualización y procesamiento de datos de frecuencia cardíaca en Python? 
		\subsection*{Criterios de selección.} Se procedió a una búsqueda sistematizada de evidencia. \\
		Artículos publicados. Artículos publicados desde el año 1990 hasta la actualidad.
		\begin{itemize}
			\item Tiene que ser en python
			\item Solo de la librería: 
		\end{itemize}
		\subsection*{Objetivos: general y específicos}
			\subsubsection{Objetivo general}
			Visualizar la frecuencia cardiaca de los usuarios mediante un interfaz a partir del lenguaje Python.
			\subsubsection{Objetivos específicos}
				\begin{itemize}
					\item Desarrollar interfaz amigable con el usuario.
					\item Leer y escribir ficheros desde Python. 
					\item Diseñar interfaz para una gráfica. 
					\item Garantizar calidad del sistema. 
					\item Desarrollo de un código legible para el programador.
					\item Crear base de datos para control de datos obtenido. 
				\end{itemize}
		\subsection*{Justificación.}
		\subsection*{Marco referencial y teórico.(citas)}
		\subsubsection{Marco teórico}
		\paragraph{¿Qué es una señal biomédica?} Una señal biomédica es cualquier tipo de señal que tiene como origen el cuerpo humano y son utilizadas para el diagnóstico e investigación médica. 
		“todas las señales usadas en el diagnostico o 
		investigación médica que se originan de alguna manera en 
		el cuerpo.” \cite{monografias}
		\paragraph{¿Qué es una interfaz?} "Conexión o frontera común entre dos aparatos o sistemas independientes, que permite el intercambio de información o la coordinación de sus funciones." \cite{rae_interfaz}
		\paragraph{¿Qué es una base de datos?} Una base de datos, es una recopilación de archivos dedicados a guardar información entre sí, es decir, que pertenecen al mismo contexto; estos datos serán usados a futuro en diferentes tareas, además, pueden ser visualizados y utilizados por más de una persona. \cite{unam}
		\paragraph{¿Qué es frontend y backend?} El front end, es la parte que va a interactuar con el usuario, en el caso de nuestro proyecto será la interfaz, el back end es la parte que contiene la infraestructura que le va a dar funcionalidad a el proyecto. \cite{FrontendBackend}
		
	\section{Metodología (propuesta).}
	\printbibliography
\begin{abstract}
\end{abstract}

\end{document}          
