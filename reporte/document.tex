\documentclass[14pt]{report}

%Para ponerlo en español:
\usepackage[T1]{fontenc}
\usepackage[spanish]{babel}

%para poder citar
\usepackage[backend=biber]{biblatex}
\bibliography{bibliografía.bib}

%Para poder añadir imágenes:
\usepackage{graphicx}

% Title Page (ya no es necesario)
\title{Proyecto: Interfaz para un filtro de señal cardiaca}
\author{Caballero Trinidad Flor Isabel. \\ Cortés Ramírez Samuel Jefte. \\ García García Ian Pablo. \\ Pérez Roldan Eduardo Alejandro. \\ Cabañas Alba Alejandro. \\ Alfonso Gamboa Rubén.}


\begin{document}

%Portada
\begin{titlepage}
	\centering
	{\includegraphics[width=0.2\textwidth]{recursos/Universidad_Veracruzana}\par}
	\vspace{1cm}
	{\bfseries\LARGE Universidad Veracruzana \par}
	\vspace{1cm}
	{\scshape\Large Ingeniería Biomédica \par}
	\vspace{3cm}
	{\scshape\Huge Interfaz para un filtro de frecuencia cardiáca. \par}
	\vspace{2cm}
	{\itshape\Large Proyecto. \par}
	\vfill
	{\Large Autores: \par}
	{\Large Caballero Trinidad Flor Isabel. \\ Cortés Ramírez Samuel Jefte. \\ García García Ian Pablo. \\ Pérez Roldan Eduardo Alejandro. \\ Cabañas Alba Alejandro. \\ Alfonso Gamboa Rubén. \par}
	\vfill
	
\end{titlepage}

%\tableofcontents
\chapter{Parcial 1}
	\section{keywords}
		\begin{itemize}
			\item Frecuencia cardaica.
			\item Interfaz.
			\item Filtro de frecuencia cardaiaca.
			\item Interfaz para un filtro.
			\item SPython.
			\item Interfaz Médica.
			\item Ingeniería Biomédica.
			\item Arduino
		\end{itemize}
	
	\section{Introducción.}
		El siguiente proyecto tiene como finalidad responder a la pregunta ¿Cómo desarrollar una interfaz que facilite la visualización y procesamiento de datos de la frecuencia cardiaca?, teniendo como principal objetivo visualizar la frecuencia cardiaca del usuario mediante una interfaz a partir del lenguaje Python. 
		
		Los temas que se abordaran y se trabajaran para la realización optima del proyecto son, desarrollar una interfaz amigable con el usuario con la finalidad de interpretar los datos de manera precisa, garantizando la funcionalidad del sistema diseñando una interfaz para una gráfica, con el lenguaje de programación Python se busca lograr leer y escribir ficheros, desarrollar un código legible para el programador y crear una base de datos para los datos recolectados, con el propósito de calcular la media aritmética para añadir funcionalidad que permita al usuario calcular su recuperación cardiaca. Este proyecto esta dirigido a personas que buscan monitorear su frecuencia cardiaca.  
		
		Se llevará a cabo el proyecto con la finalidad de adquirir conocimiento sobre la creación de una interfaz en el lenguaje de programación Python, con la ayuda de una practica anterior relacionada a la Ingeniería Biomédica, se obtendrán lecturas de medición para la realización de prueba predictiva en índices para medir la salud cardiaca, se diseñara una versión base de una interfaz para el almacenamiento de señales biomédicas, con la finalidad de utilizarlas en un futuro y poder manipularlas, también se trabajara en conjunto con compañeros de la carrera de Instrumentación Electrónica, para así reforzar el trabajo en equipo y crear un front-end y un banck-end. 
		
		Para filtrar la búsqueda de la información necesaria para poder realizar el proyecto, se procedió a una búsqueda sistematizada de evidencia siguiendo los criterios de inclusión, por ejemplo, artículos, libros, tesis, publicadas desde el año 1990 hasta la actualidad, información relacionada a como crear una interfaz en el lenguaje de programación Python, información del funcionamiento y características de la frecuencia cardiaca. 
		
		En esta sección se elabora una revisión bibliográfica de los conceptos generales a partir de los cuales se sustenta este proyecto, los conceptos a considerar son: señales biomédicas, interfaz, base de datos, front-end, back-end, lenguaje de programación Python, frecuencia cardiaca. 	
	\section{Metodología.}
	Se utilizará el programa Python, con la librería “TkInter” se diseñarán los algoritmos para el despliegue de las gráficas de pulso cardiaco. \\
	Para la obtención de las señales cardiacas se utilizará un sensor” Max 30100” y una placa arduino uno con el cual se medirá el pulso cardiaco de una persona para posteriormente guardarlo en un archivo “.txt” y en un archivo Excel, dicho archivo pasará por un filtro realizado por nuestros compañeros de instrumentación electrónica para así entregarnos una señal legible y más limpia, con esto podremos hacer la interfaz en Python.
	\printbibliography
%\begin{abstract}
%\end{abstract}

\end{document}          
